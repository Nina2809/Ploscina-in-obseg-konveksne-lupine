% Options for packages loaded elsewhere
\PassOptionsToPackage{unicode}{hyperref}
\PassOptionsToPackage{hyphens}{url}
%
\documentclass[
]{article}
\usepackage{amsmath,amssymb}
\usepackage{lmodern}
\usepackage{ifxetex,ifluatex}
\ifnum 0\ifxetex 1\fi\ifluatex 1\fi=0 % if pdftex
  \usepackage[T1]{fontenc}
  \usepackage[utf8]{inputenc}
  \usepackage{textcomp} % provide euro and other symbols
\else % if luatex or xetex
  \usepackage{unicode-math}
  \defaultfontfeatures{Scale=MatchLowercase}
  \defaultfontfeatures[\rmfamily]{Ligatures=TeX,Scale=1}
\fi
% Use upquote if available, for straight quotes in verbatim environments
\IfFileExists{upquote.sty}{\usepackage{upquote}}{}
\IfFileExists{microtype.sty}{% use microtype if available
  \usepackage[]{microtype}
  \UseMicrotypeSet[protrusion]{basicmath} % disable protrusion for tt fonts
}{}
\makeatletter
\@ifundefined{KOMAClassName}{% if non-KOMA class
  \IfFileExists{parskip.sty}{%
    \usepackage{parskip}
  }{% else
    \setlength{\parindent}{0pt}
    \setlength{\parskip}{6pt plus 2pt minus 1pt}}
}{% if KOMA class
  \KOMAoptions{parskip=half}}
\makeatother
\usepackage{xcolor}
\IfFileExists{xurl.sty}{\usepackage{xurl}}{} % add URL line breaks if available
\IfFileExists{bookmark.sty}{\usepackage{bookmark}}{\usepackage{hyperref}}
\hypersetup{
  pdftitle={Poročilo projekta - Aproksimacija ploščine in obsega konveksne lupine},
  pdfauthor={Nina Velkavrh in Ajda Majhenič},
  hidelinks,
  pdfcreator={LaTeX via pandoc}}
\urlstyle{same} % disable monospaced font for URLs
\usepackage[margin=1in]{geometry}
\usepackage{graphicx}
\makeatletter
\def\maxwidth{\ifdim\Gin@nat@width>\linewidth\linewidth\else\Gin@nat@width\fi}
\def\maxheight{\ifdim\Gin@nat@height>\textheight\textheight\else\Gin@nat@height\fi}
\makeatother
% Scale images if necessary, so that they will not overflow the page
% margins by default, and it is still possible to overwrite the defaults
% using explicit options in \includegraphics[width, height, ...]{}
\setkeys{Gin}{width=\maxwidth,height=\maxheight,keepaspectratio}
% Set default figure placement to htbp
\makeatletter
\def\fps@figure{htbp}
\makeatother
\setlength{\emergencystretch}{3em} % prevent overfull lines
\providecommand{\tightlist}{%
  \setlength{\itemsep}{0pt}\setlength{\parskip}{0pt}}
\setcounter{secnumdepth}{-\maxdimen} % remove section numbering
\ifluatex
  \usepackage{selnolig}  % disable illegal ligatures
\fi

\title{Poročilo projekta - Aproksimacija ploščine in obsega konveksne
lupine}
\author{Nina Velkavrh in Ajda Majhenič}
\date{26.11.2021}

\begin{document}
\maketitle

\hypertarget{uvod---definicija-konveksne-lupine}{%
\subsubsection{Uvod - Definicija konveksne
lupine}\label{uvod---definicija-konveksne-lupine}}

Podmnožica \(T\) v \(\mathbb{R}^2\) je konveksna, če lahko poljubni
točki \(A\) in \(B\) v \(T\) povežemo z daljico, ki je v celoti
vsebovana v množici \(T\).

Naj bo \(S\) končna množica točk v ravnini. Konveksna lupina je
najmanjša konveksna množica, ki vsebuje \(S\).

Z drugimi besedami mejo konveksne lupine tvori konveksni poligon,
katerega ogljišča so točke množice \(S\), robovi pa so odseki, ki
povezujejo pare točk množice \(S\). Označimo konveksni mnogokotnih s
\(CH(S)\).

\hypertarget{opis-problema}{%
\subsubsection{Opis problema}\label{opis-problema}}

V projektni nalogi sva se ukvarjali z zelo pogostim problemom geometrije
in sicer konstruiranje konveksne lupine s končno množico točk v ravnini.
Osredotočili sva se predvsem na izračun njene ploščine in obsega.
Predpostavljali sva, da imamo konveksno lupino množice \(P\), znotraj
katere je \(n\) točk. Sprva sva implementirali znan Jarvisovega
algoritem za določitev konveksne lupine množice S z n točkami, nato sva
izračunali eksakten obseg in ploščino. V glavnem delu projekta sva
modelirali 2 metodi za njuno aproksimacijo. Dobljene ploščine in obsege,
sva med sabo primerjali in zelo hitro ugotovili, da je 2. metoda precej
slabša kot 1., saj je le ta vračala precej večje napake kot 1..
Doblejene rezultate sva primerjali tudi z dejanskimi vrednostmi obsega
in ploščine. Zanimalo naju je koliko elementov v vzorcu (podmnožici
množice S) potrebujeva, da dobiva približek z natančnostjo 90\% , 99\%
in 99,9\% ter ali ima pri tem število točk kakšno vlogo.

\hypertarget{opis-algoritma-za-doloux10ditev-konveksne-lupine-mnoux17eice-s-z-n-toux10dkami}{%
\subsubsection{Opis algoritma za določitev konveksne lupine množice S z
n
točkami}\label{opis-algoritma-za-doloux10ditev-konveksne-lupine-mnoux17eice-s-z-n-toux10dkami}}

Ob začetku najnega projekta sva v najin program implementirali zelo znan
in intuitive algoritem za določanje konveksne lupine imenovan Jarvisov
algoritem. Ta algoritem je poznan tudi po imenu algoritem zavijanja
daril, saj se premika od enega oglišča konveksne lupine do drugega kot,
da ovijamo kos papirja okoli množice točk.

\newline Uporabljen algoritem deluje po naslednjem postopku:

\begin{itemize}
\item
  Začnemo s tem, da najdemo točko \(p_1\) iz množice \(S\), ki je
  oglišče konveksne lupine. Za \(p_1\) vzamemo najnižjo točko množice
  \(S\). V kolikor množica \(S\) vsebuje več točk z minimalno \(y\)
  koordinato, vzamemo najbolj levo točko med njimi (točko z najmanjšo
  \(x\) koordinato).
\item
  Predpostavili smo, da je \(p_1\) oglišče konveksne lupine, posledično
  mora obstajati tudi taka točka \(p_2\) iz množice \(S\), da je tudi
  \(p_2\) oglišče konveksne lupine in da je \(p_1p_2\) rob konveksne
  lupine. Točki, ki bi ustrezali omenjenemu pogoju sta dve. Za \(p_2\)
  izberemo tisto točko, ki naredi \(p_1p_2\) za rob, ki se bo gibal v
  smer urinega kazalca.
\item
  Kako najdemo \(p_2\)?
\end{itemize}

\newline Naj bo \(l_1\) vodoravna premica skozi \(p_1\). Potem je
\(p_2\) prva točka, ki jo ``zadanemo'', če vrtimo \(l_1\) okrog \(p_1\)
v smer urinega kazalca. Naj bo \(\alpha _q\) kot med \(l_1\) in premico
\(p_1q\) za vsak \(q \in S \setminus \lbrace p_1 \rbrace\). Potem je
\(\alpha _q\) kot po katerem moramo vrteti \(l_1\) okrog \(p_1\) v smeri
urinega kazalca, dokler ne ``zadanemo'' \(p_1q\). Upoštevamo, da je
\(0 \leq \alpha _q \leq 2\pi\). Torej je \(p_2\) tista točka množice
\(S \setminus \lbrace p_1 \rbrace\), za katero je ta kot minimalen. Če
obstaja več točk, za katere je kot \(\alpha\) minimalen, potem je
\(p_2\) tista izmed teh točk, ki ima maksimalno razdaljo od \(p_1\).

\newline Torej lahko za dano točko \(p_1\) najdemo naslednje oglišče
konveksne lupine \(p_2\) tako, da preverimo vse točke
\(q \in S \setminus \lbrace p_1 \rbrace\) in izberemo tisto, za katero
je kot \(\alpha _q\) minimalen.

\begin{itemize}
\tightlist
\item
  Nadaljujemo seveda po enakem postopku: Naj bo \(l_2\) premica skozi
  točki \(p_1\) in \(p_2\). Za vsako točko
  \(q \in S \setminus \lbrace p_2 \rbrace\) naj bo \(\alpha _q\) kot med
  \(l_2\) in premico \(p_2q\). (Sedaj je \(\alpha _q\) kot po katerem
  moramo vrteti \(l_2\) okrog \(p_2\) proti smeri urnega kazalca, dokler
  ne ``zadanemo'' \(p_2q\).) Potem je \(p_3\) - naslednje oglišče
  konveksne lupine - tista točka, katere kot \(\alpha\) je minimalen.
\end{itemize}

Nadaljujemo z izračunavanjem oglišč \(p_4\), \(p_5\), \ldots, dokler se
ne vrnemo v \(p_1\). Bolj natančno: Če \(p_{h+1} = p_1\), smo končali in
imamo seznam oglišč kompleksne lupine \((p_1, p_2, ... , p_h)\).

\newline Po imlplementaciji zgoraj opisanega algoritma sva definirali
obseg in ploščino. Obseg sva izračunali tako, da sva računali razdalje
med točkami konveksne lupine in vrednosti med seboj seštevali. Ploščino
pa sva izračunali tako, da sva implementirali znano formulo za določitev
ploščino poljubnega lika imenovano Shoelace formula.

Seveda sva za določitev konveksne lupine, izračun ploščine in obsega
potrebovali množico S z n točkami. Zato sva definirali naključno
množico, ki sva ji določili parametre a, b in qty. V implementaciji
Jarvisovega algoritma sva že definirali točko, zato sva tu potrebovali
le še naključno izbiro za koordinato x in y. Parameter a določi, da
koordinate x izbiramo iz intervala {[}o,a{]}, parameter b pa, da
koordinato y izbiramo iz območja {[}0,b{]}. Parameter qty določi koliko
naključnih točk iz območja {[}0,a{]}x{[}0,b{]} želimo. Sicer sva tu
definirali 2 parametra za določitev območja, a sva kasneje v algoritmu
za aproksimacijo ploščine in obsega uporabljali le še kvadrate, torej bi
lahko definirali tudi samo parameter a.

Kasneje sva se lotili še glavnega dela projektne naloge. Za algoritem,
ki bi lahko aproksimiral ploščino in obseg območja konveksne lupine sva
imeli 2 ideji. Oba algoritma sta spodaj podrobneje opisana.

\hypertarget{opis-1.-metode}{%
\subsubsection{Opis 1. metode}\label{opis-1.-metode}}

Pri prvi metodi začnemo s praznim seznamom pravokotnikov. Privzamemo, da
je vrednost a/m dolžina, vrednost b/m pa višina. To nam območje {[}0,
a{]}x{[}0, b{]} razdeli na pravokotnike (kvadrate) velikosti m x m. Iz
vsakega dobljenega pravokotnika izberemo kvečjemu eno točko (če je v
pravokotniku več točk množice S, naključno izberemo eno izmed njih, če
pa v njem ni nobene točke iz množice S, potem ta pravokotnik
»preskočimo«). Izbrane točke zberemo v seznam, ki sedaj predstavlja
podmnožico osnovne množice S in v najnem alogritmu en vzorec. Dobljenemu
vzorcu (podmnožici) določimo konveksno lupino, njeno ploščino in obseg.

\hypertarget{opis-2.-metode}{%
\subsubsection{Opis 2. metode}\label{opis-2.-metode}}

\hypertarget{statistiux10dno-modeliranje-v-programu-r}{%
\subsubsection{Statistično modeliranje v programu
r}\label{statistiux10dno-modeliranje-v-programu-r}}

\hypertarget{obdelava-podatkov-v-programu-r}{%
\paragraph{Obdelava podatkov v programu
r}\label{obdelava-podatkov-v-programu-r}}

Podakte sva uvozili v program r in jih statistično obdelali. Med
analiziranjem rezultatov naju je zanimalo:

\begin{itemize}
\tightlist
\item
  Koliko je povprečni delež točk v vzorcu 1. modela, ki ga potrebujeva
  za 90\% , 99\% in 99,9\% natančnost pri različnih velikostih osnovne
  množice S,
\item
  kako na rezultate modela 1.vpliva različna razdelitev območja pri
  fiksnem številu točk,
\item
  kako se razlikujejo rezultati posamezne metode glede na različno
  število točk v množici S,
\item
  ali in kdaj je mogoče tudi z 2. metodo dobiti 90\% , 99\% in 99,9\%
  natančnost,
\end{itemize}

\hypertarget{rezultati}{%
\subsubsection{Rezultati:}\label{rezultati}}

\hypertarget{povpreux10dni-deleux17e-toux10dk-za-90-99-in-999-natanux10dnost-1.-modela-pri-izraux10dunu-ploux161ux10dine-in-obsega-pri-razliux10dnih-moux10deh-mnoux17eice-s}{%
\paragraph{Povprečni delež točk za 90\% , 99\% in 99,9\% natančnost 1.
modela pri izračunu ploščine in obsega pri različnih močeh množice
S}\label{povpreux10dni-deleux17e-toux10dk-za-90-99-in-999-natanux10dnost-1.-modela-pri-izraux10dunu-ploux161ux10dine-in-obsega-pri-razliux10dnih-moux10deh-mnoux17eice-s}}

Ob začetku analize podatkov sva si odgovorili na vprašanje kakšen delež
točk je potreben za 90\% , 99\% in 99,9\% natančnost 1. modela. Program
je zgeneriral točke na območju {[}0,10{]} x {[}0,10{]}, pri 100 delitvah
enega intervala - torej mreža 10000 kvadratov, pri različnih močeh
množice S. Sprva sva podatke generirali tako, da se je za vsako moč
množice S algoritem ponovil stokrat, ker sva pri takem številu ponovitev
ugotovili, da obstajajo odstopanja pri deležih, sva se odločili, da
algoritem poženeva raje tisočkrat za vsako moč množice S. Generiranje
podatkov je sicer trajalo kar nekaj časa, a sva kasneje z analizo
ugotovili, da je tovrstn način bolj reprezentativen. Rezultate zadnjega
poskusa prikazujeta spodja grafa. Opazi se, da funkciji nimata
nenavadnih odstopanj. To, da ne opazimo odstopanj se zdi logično, saj se
pri večjem številu iteracij izračunana povprečna vrednost deležev točk,
ki je potrebna za željeno natančnost bolj približuje pravi povprečni
vrednosti. Iz spodnjih grafov je moč opaziti, da se povprečni delež točk
potrebnih za željeno natančnost, zmanjšuje z večanjem moči množice S.
Videno je verjetno posledica tega, da je območje glede na količino točk
v večjih množicah precej majhno in so posledično točke zelo gosto
razporejene po območju. Kot omenjeno model 1 iz vsakega dela mreže na
katero je razdeljen kvadrat {[}0,10{]} x {[}0,10{]}, vzame le 1 točko
oziroma nobene, če celica nima točke množice S. Posledica tega je, da so
neizbrane točke verjetno zelo blizu izbranim in se rezultati
aproksimacije ne razlikujejo bistveno od eksaktnih vrednosti ploščin in
obsega.

\newline Grafa prikazujeta potrebne deleže točk za željene natančnosti
aproksimacije obsega in ploščine. Vidimo lahko, da so potrebni deleži za
obseg le nekaj odstotnih točk manj kot potrebni deleži za ploščino.

\includegraphics[width=0.5\linewidth]{Koncno_porocilo_files/figure-latex/graf_razlicnih_delezev-1}
\includegraphics[width=0.5\linewidth]{Koncno_porocilo_files/figure-latex/graf_razlicnih_delezev-2}

\hypertarget{vpliv-razdelitve-obmoux10dja-na-rezultate-1.-metode}{%
\paragraph{Vpliv razdelitve območja na rezultate 1.
metode}\label{vpliv-razdelitve-obmoux10dja-na-rezultate-1.-metode}}

dodati morva graf

\hypertarget{primerjava-rezultatov-1.-in-2.-metode-pri-razliux10dnih-moux10deh-mnoux17eice-s}{%
\paragraph{Primerjava rezultatov 1. in 2. metode pri različnih močeh
množice
S}\label{primerjava-rezultatov-1.-in-2.-metode-pri-razliux10dnih-moux10deh-mnoux17eice-s}}

Spodnja grafa prikazujeta povprečne rezultate obeh metod pri tisočih
ponovitvah metod za vako moč množice S. Tudi pri spodnjih grafih sva
sprva poskusili algoritem ponoviti le stokrat za vsako moč množice, a je
prišlo tudi tu do precejšnih odstopanj. Razlog odstopanj je podoben kot
pri izračunu potrebnih deležev. Torej, da se pri večjem številu
ponovitev modela izračunana povprečna vrednost napake za vsako moč
množice bolj približuje pravi napaki. Opazimo lahko, da so napake 2.
metode bistveno večje, kot napake 1. metode. Razlog za to je, da 2.
metoda vzame za polmer kroga točko, ki je najbolj oddaljena od težišča
(povprečja koordinat točk), ta razdalja je lahko v primeru, da je večina
točk zgoščena okoli neke vrednosti in le majhen delež točk popolnoma
drugje, precej velika. Ta majhen delež točk, ki se nahaja stran od
večine ne prispeva veliko pri izračunu povprečja koordinat, zato se
težišče nahaja bližje območju kjer je večji delež točk množice S.
Posledično bo krog vzel precej večje območje, kot je konveksna lupina te
množice in bojo napake aproksimacije zelo velike. Podoben razmislek, kot
smo ga uporabili pri razlagi, zakaj je za večje množice potreben manjši
delež točk za željeno točnosti, lahko uporabimo tudi tu, ko opazimo, da
se napake 2. metode manjšajo z večjim številom točk množice S. Na
območju {[}0,10{]} x {[}0,10{]} je namreč zelo verjetno, da bojo točke
večjih množic bolj enakomerno porazdeljene po območju in bo krog
posledično manj odstopal od dejanske konveksne lupine. Zanimiv podatek,
ki sva ga opazili je, da na neki toči napaka ne pade pod 55\% za
ploščino in 10\% za obseg. Razlog tega je verjetno ta, da se z večanjem
vzorca povečuje verjetnost, da bo najbolj oddaljena točka od težišča
blizu roba, težišče pa blizu sredini kvadrata in bo s tem krog očrtan
krog kvadrata.

\newline Pri grafu, ki prikazuje povprečne napake 1. metode lahko
opazimo, da so napake med 0 in 0.15, a se z večanjem vzorca povečujejo.
Slednje se dogaja zato, ker je pri izbiri točk pri manjših močeh množice
bolj verjetno, da so izbrane točke ravno točke, ki sestavljajo konveksno
lupino. Medtem, ko gostota razporeditve točk večjih množic S vpliva na
to, da je verjetnost, da izberemo ravno točke ovojnice manjša.

\includegraphics[width=0.5\linewidth]{Koncno_porocilo_files/figure-latex/graf_napak_2.metode-1}
\includegraphics[width=0.5\linewidth]{Koncno_porocilo_files/figure-latex/graf_napak_2.metode-2}

\hypertarget{metoda-in-njena-uspeux161nost}{%
\paragraph{2. metoda in njena
uspešnost}\label{metoda-in-njena-uspeux161nost}}

\begin{center}\includegraphics{Koncno_porocilo_files/figure-latex/graf_napak_za_krog-1} \end{center}

\hypertarget{zakljuux10dek-in-moux17ene-izboljux161ave}{%
\subsection{Zaključek in možne
izboljšave}\label{zakljuux10dek-in-moux17ene-izboljux161ave}}

\end{document}
